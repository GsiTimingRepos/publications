\section{White Rabbit PTP}
\label{sec:wr}

%\subsection{White Rabbit Synchronization}
WR reaches sub-nanosecond synchronization with picosecond jitter, basically, 
characterizing the asymmetries of the link, and using a clock loopback
technique for tracking the phase shift between the master reference clock
signal and the clock signal of the slave. In order to gather and
accomplish such a measurement WR extends PTP, WRPTP~\cite{biblio:ispcs_m}. 
\figurename~\ref{fig:wr_ptp} shows the WRPTP flow of messages and the
integration with standard PTP once is established.

Below is the summary of the steps, measurement and hardware support needed for the 
WR synchronization between two WR devices. A thorough description is presented 
in ~\cite{biblio:tomas} and ~\cite{biblio:wrptp}.

\subsubsection{WR Discovery and Syntonization}

A WR clock initiates WRPTP with an announcement message in order to discover
other WR devices. If a WR device has been discovered, the master will initiate the 
frequency lock procedure. WRPTP uses SyncE to distribute a common frequency throughout 
the network, the WR Slaves are frequency locked to the WR Master. 

\subsubsection{Asymmetry Calibration}

Once the slave is locked to the master, the WR devices initiate to calibrate the 
asymmetries in the common optical link taking in consideration:

\begin{itemize}
    \item Fixed delays due to transmission circuitry
    \item Asymmetry of the optical transceivers and PHYs 
    \item Asymmetry of the propagation delay in the fiber caused by the chromatic dispersion
\end{itemize}

\subsubsection{Coarse and Fine Delay Measurement}
\label{sec:delay_ms}
After the devices are calibrated, a first delay measurement, \textit{Coarse
Measurement}, is issued using the delay request-respond (DRR) mechanism. 
The next step towards the synchronization requires the measurement of the phase shift
between a reference (master) clock signal and the loopback clock signal
of the slave. For this purpose, WR implements Digital Dual Mixer Time Difference
(DDMTD) ~\cite{biblio:ddmtd} phase detection. The DDMTD measures the \textit{round trip phase
shift}, $phase_{mm}$.

With the $phase_{mm}$, the delay round trip, $delay_{mm}$ is calculated. The $phase_{s}$
is the phase shift of the clock adjustment (see \figurename~\ref{fig:wr_time_stamp})  
derived from the $offset_{ms}$. Using the $phase_{mm}$ and $phase_{s}$, 
the timestamps on ingress ports can be enhanced,
$t_{2p}$ and  $t_{4p}$, using a decision algorithm described in ~\cite{biblio:tomas}.
Only timestamps on ingress ports need to be enhanced, since they are generated 
asynchronously to the reference clock domain. The \textit{Fine Delay Measurement}, round trip,
is calculated as follows, using the enhanced timestamps :

\begin{equation}
  \label{eq:round_trip}
    delay_{mm} = (t_{4p} - t_1) - (t_3 - t_{2p})
\end{equation}

\figurename~\ref{fig:wr_ptp} shows how after the initial WRPTP synchronization, 
a DRR or peer delay (PD) mechanism produces the timestamps, $t_{1}$, $t_{2p}$,
$t_{3}$ and $t_{4p}$ (in the case of PD, also
$t_{5}$ and $t_{6p})$.

\subsubsection{Synchronization}
In order to finish the synchronization, the offset between both clocks must be
calculated, $offset_{ms}$. 
The round trip can be expressed as a function of the  delay master to slave, $\sigma _{ms}$ , slave to
master $\sigma _{sm}$ and the sum of the fixed delays, $\Delta$ , obtained during the calibration.

\begin{equation}
  \label{eq:round_trip_2}
    delay_{mm} = \Delta + \sigma _{ms} + \sigma _{sm}
\end{equation}

The ratio between single delays is proportional to the asymmetry of the speed of
the different wavelengths due to the chromatic dispersion in the fiber optic
link:

\begin{equation}
    \label{eq:disp}
    (\alpha-1) = \frac{\sigma_{ms}}{\sigma_{sm}}
\end{equation}

\begin{figure}[!t]
\centering
\includegraphics[scale=0.26]{fig/time_stamps_tc.png}
\caption{WR Loopback Clock, Phase Detection and Enhanced Time Stamps}
\label{fig:wr_time_stamp}
\end{figure}


Combining equations (\ref{eq:round_trip}), (\ref{eq:round_trip_2}) and
(\ref{eq:disp}), the delay master to slave and $offset_{MS}$:

\begin{equation}
    \label{eq:delayms}
     delay_{ms} = \frac{1+ \alpha}{2+ \alpha}(delay_{mm} - \Delta)+ \Delta_{txm} + \Delta_{rxs} 
\end{equation}

\begin{equation}
    \label{eq:offsetms}
    offset_{ms} = t_{1} - t_{2p} - delay_{ms}
\end{equation}

\begin{figure}[!t]
\centering
\includegraphics[scale=0.25]{fig/wr_ptp.png}
\caption{WR PTP Message Flow and PTP}
\label{fig:wr_ptp}
\end{figure}



\documentclass{JAC2003}

%%
%%  This file was updated in April 2009 by J. Poole to be in line with Word tempaltes
%%
%%  Use \documentclass[boxit]{JAC2003}
%%  to draw a frame with the correct margins on the output.
%%
%%  Use \documentclass[acus]{JAC2003}
%%  for US letter paper layout
%%

\usepackage{graphicx}
\usepackage{booktabs}

%%
%%   VARIABLE HEIGHT FOR THE TITLE BOX (default 35mm)
%%

\setlength{\titleblockheight}{27mm}

\begin{document}
\title{FAIR Timing System Developments Based on White Rabbit}

\author{C. Prados, R. B{\"a}r, D. Beck, J. Hoffmann, N. Kurz, S. Rauch, W. Terpstra, M. \\
Zweig (GSI, Darmstadt), M. Kreider (GSI, Darmstadt; Glyndwr University, \\
Wrexham)}
\maketitle


\begin{abstract}
 A new timing system based on White Rabbit (WR) is being developed for the
 upcoming FAIR facility at GSI, in collaboration with CERN, other institutes and
 industry partners. The timing system is responsible for the synchronization of
 nodes with nanosecond accuracy and distribution of timing events, which
 allows for real-time control of the accelerator equipment. WR is a fully
 deterministic Ethernet-based network for general data transfer and
 synchronization, which is based on Synchronous Ethernet and PTP. The ongoing
 development at GSI aims for a miniature timing system, which is part of a
 control system of a proton source, that will be used at one of the accelerators
 at FAIR. Such a timing system consists of a Data Master generating timing
 messages, which are forwarded by a WR switch to a handful of timing receivers.
 The next step is an enhancement of the robustness, reliability and scalability
 of the system. These features will be integrated in the forthcoming CRYRING
 control system in GSI. CRYRING serves as a prototype and testing ground for the
 final control system for FAIR. The contribution presents the overall design and
 status of the timing system development.
\end{abstract}


\section{INTRODUCTION}

The GMT triggers and synchronizes accelerator equipment accordingly 
to the accelerator cycles~\cite{fair_rep}. Cycle lengths range 
from 20 ms (present UNILAC), several seconds (synchrotrons SIS18 
and SIS100/300) to several hours (storage rings)~\cite{gmt}. The beam production chain is an important concept in
accelerator control systems. It describes the production of a beam
from an ion source through the accelerators to a target. Properties of such a beam production 
chain include the ion type (from protons to uranium), energy, intensity, focus
and emittance and other parameters at the final destination. This information is
conveyed to the GMT, which 
has an integral view on the tightly synchronized accelerators and beam transfer
sections. The GMT must take into account the execution of several beam production 
chains in the accelerator complex at the same time. For each part of the machine, switching 
between different beam production chains will be possible between cycles, which implies a 
high degree of true parallel operation.

The GMT is made of an interconnected network of Timing Receiver Nodes (TRN) and
special nodes so-called Masters. The TRN are devices synchronized to the Timing
Master, source of time and frequency for all the network, and they are also the
receivers of timing messages from the Data Master. The interconnection is
established using WR Switches, which are responsible of the propagation of the
synchronization and timing messages to the TRN.

In  about  four years,  first  beam will  be  injected  into the  FAIR
accelerator complex. By then,  the General Machine Timing system
must support about 2000  Front End Controllers (FEC) with integrated TRN. 
The main tasks of the GMT are distribution 
of  timing messages, clocks  and timestamps.  Timing messages  synchronize actions of
the accelerator in hard real-time on a level of about 1 ns, even if
components are a few kilometers apart. Clock and timestamps
distribution, inherently  provided by White  Rabbit PTP, will  be used
for  correlating  data  acquired  at distinct  places  throughout  the
facility. 

The GMT  is linked to other  systems.  It must react  with upper bound
latency  on  external  signals  like interlock  signals~\cite{interlock}  by  executing
predefined alternatives in the  schedule. All systems connected to the
timing  system  depend on  it's  high  availability.  Distribution  of
timing messages, clocks and timestamps  must be guaranteed for commissioning and
testing even when the accelerator  does not produce beam. As a failure
in synchronizing equipment  may led to loss of  beam, the distribution
of timing messages must be robust -  at most one timing messages per year
may  be lost.  Furthermore, critical  components  of the  GMT must  be
implemented redundantly.

The  GMT not only  serves for  the operation  of the  FAIR accelerator
machines. Also Data AcQuisition (DAQ) systems of experiments will link
to  the  GMT  for   correlating  experimental  data  with  accelerator
actions. Moreover, a connection to the GMT allows for time stamping of
data in case of globally triggered DAQ systems, fan-out of time stamps
for  free running  DAQ systems  and finally  provides  distribution of
precise synchronous clock signals.


\section{DESIGN}

GMT is a decentralized design, Fig.~\ref{network}, based on a network where every kind of device
is responsible of carrying out an specific tasks. The Clock Master is source of time and frequency for the network.
The Clock Master creates, schedules and sends timing messages to control the accelerator
and the WR Network (WRN) provides synchronization propagation and transports the data (e.g timing messages) to the TRN. 
The hardware and software design of this devices is being developed in GSI (e.g TRNs) and in
collaboration with other institutes and companies (e.g WR Switch). 

\subsection{White Rabbit}

White Rabbit (WR~\cite{wr}) is a protocol developed to
synchronize nodes in a packet-based network with sub-ns accuracy. WR is based 
on existing standards: Ethernet (IEEE 802.3~\cite{internet}),
Synchronous Ethernet (SyncE~\cite{sync}) and PTP~\cite{ptp}. 
A WR network offers low-latency, deterministic packet delivery and network-wide, 
transparent, high-accuracy timing distribution. 

The timing network at SIS/FAIR is a White Rabbit network interconnecting TRNs 
using WR Switches and fiber optic links.

\begin{figure}[htb]
   \centering
   \includegraphics*[scale=0.5]{THPPC092f1.eps}
   \caption{General Machine Time}
   \label{network}
\end{figure}


\subsection{Data Master}

The Data Master (DM) for the FAIR accelerator
deals with various different tasks. The first step is to
take machine commands from the LHC Software Architecture (LSA) and convert these 
into sequence programs, depicting beam production chains~\cite{dm}.
The corresponding actions that are to be taken are preprogrammed into the timing endpoints. The DM will run
a multiple of these sequence programs in parallel and
each is generating timing messages containing action
IDs and their execution times. These are scheduled and
handed over to a dispatcher in order to be sent over the timing
network. The sequence programs will communicate with each other to
resolve dependencies in beam production chains and
are able to take external signals and interlocks into
account.

The DM possesses a high end CPU running an
OS for easy interfacing to the control system, compatibility to 
FAIRs standard libraries and raw processing power as well as a 
Field Programmable Gate Array (FPGA) for parallelism, deterministic 
behavior and ultra low IO latency.

As Fig.~\ref{fatima} shows, the hardware and functional, see design of the DM is subdivided into
parts:

\textit{CPU - API Block} \\
Its CPU is fed by the LHC Software
Architecture (LSA) with machine parameters derived from
physical requirements for beam production chains. These
parameters are converted into sequence programs,
compiled and uploaded to the FPGA of the DM.

\textit{FPGA - SoftCPU Cluster} \\
These programs are run in
parallel on SoftCPU macros residing in the FPGA. They
deal with sending out timing messages paired with an execution time to WR nodes, 
reacting to interlocks and mutual synchronisation. At the moment, 32 of these Soft CPUs are
foreseen in the DM, able to carry out 32 tasks full parallel with IO service times of less than 50ns.

\textit{FPGA - Timing Message Concentrator} \\
An timing message concentrator macro in the DM will act as a bridge to the WR
network. Its primary functions are aggregation of messages
into Ethernet Frames and to schedule transmission of these
timning messages over the timing network so they arrive on
time at the respective nodes. 

The first version of the DM is using hardware cores for specific tasks.
Programmable hardware timer interrupts were added to the scheduler.
The dispatcher core and the etherbone protocol encoder are now also hardware
macros. It also uses three soft CPUs and as many sequence programs while now
employing message signalled interrupts for internal communication.
This version will still use fixed sequence programs with parameters rather
than code auto-generated by the LSA system. It is currently under development and
shortly awaiting final testing. It is meant to control the CRYRING accelerator
as a testbed for the final FAIR facility. The third generation of the DM should scale up to 32 soft CPUs and
will use sequence programs automatically generated by the LSA core. After being
tested on the CRYRING, it will be used to control GSI's SIS18 synchrotron and the UNILAC linear
accelerator. If the design and implementation are proven adequate, the final version will
be used to control the future FAIR facility.

\begin{figure}[htb]
   \centering
   \includegraphics*[scale=0.33]{THPPC092f2.eps}
   \caption{Data Master Design}
   \label{fatima}
\end{figure}


\subsection{Clock Master}
Timing is distributed in the WRN from a switch called Clock Master (CM) to all
the other TRNs/WR switches in the network. All the devices in the WRN lock their frequency 
and adjust their local clocks to that of the CM. The CM is basically a WR Switch
configured as Master Clock in the PTP protocol. It uses 10 Mhz and PPS signal from
a GPS for synthonization and the UTC Time for the synchronization of the
network. The Fig.~\ref{network} shows that the CM is on top of the network
connected to the first layer of WR switches. In a WR test network deployed in
GSI, a WR Switch v3 is already synchronizing prototype TRNs scattered in the
campus of GSI.  

\subsection{Management Master}

At startup, the WR networking devices need essential configuration parameters (e.g. ip
address). During operation, the network has to be constantly monitored in
order to prevent, identify and solve malfunctions or breakdowns of devices. 
These tasks are carried out by specialized software (e.g DHCP) installed in the
so-called Management Master (MM). 

In the current WRN the MM is also a gateway between the corporative, timing and management 
network. All the WR switches are connected both to management and timing
network, while the WR nodes are only to the timing network. 

Currently the MM is serving ip addresses to the management port of switches and WR
nodes using BOOTP~\cite{bootp}. Information about the status of the network (e.g.
link up/down, synchronization status etc...) is gathered in the MM using a
Distributed Information Management
tool. Finally, the WR Switches can boot using a NFS server in the MM for testing
purposes. 

In the future the MM will offer advance management and monitoring capabilities
using the SNMP~\cite{snmp} and sFlow~\cite{sflow}. These tools will allow to the 
WR network manager to anticipate networks problems and maintain the reliability
and robustness required for the timing system.


\subsection{Timing Receiver Nodes}

FAIR requires multiple form factor variants of timing receiver nodes.
In order to reduce the maintenance effort,
all of our different FPGAs utilize a common system on chip (SoC)
design Fig.~\ref{soc}.
This design is centered around the Wishbone bus system,
which combines the standard timing receiver functionality with 
the form factor specific bus interfaces.
The standard functionality included in every timing receiver
consists of White Rabbit, an Event-Condition-Action (ECA) scheduler,
and a timestamp latch unit (TLU).
The form factor specific interfaces fall into three categories.
First, master interfaces for controlling the timing receiver,
such as PCIe, VME, USB, and Ethernet.
Second, slave interfaces for controlling off-chip resources,
such as DDR3, SRAM, flash, daughter boards, and displays.
Third, raw IO interfaces suitable for capturing signals to 
timestamps using the TLU or generating high precision timing
signals using the ECA.
These last interfaces are generally LEMO, LVDS, or high density connectors.

\begin{figure}[htb]
   \centering
   \includegraphics*[scale=0.5]{THPPC092f3.eps}
   \caption{TRN Common System on Chip Design}
   \label{soc}
\end{figure}


The ECA unit is a gateware component for producing control
signals at preprogrammed times.
The idea is to split high-level timing command, namely events,
which should be carried out by multiple devices at a given time,
from the actions an individual timing receiver must take.
To this end, 
after attaching a timing receiver to a controlled device,
one programs the ECA's condition table.
This rules in this table 
specify which events from the DM
require action by the timing receiver.
Furthermore, the rules may include a time offset to compensate 
for local delays due to the attached table length or delays
inherent to the controlled device.

As a concrete example, 
one could program the ECA to respond to ramp events
by outputting new set values to the magnet's power supply.
When the DM broadcasts a timing message to apply in 200us,
all the timing receivers process this request.
Those timing receivers which control magnets on the ring
recognize this event requires their action.
They calculate when they must power their magnets to achieve
the 200us target and schedule an action for that time.
When the time arrives, the ECA executes the required action,
accurate to 8ns.
In the future,
we intend to leverage Altera's PLL phase shifting technology 
to reach 100ps accuracy.

Similar to the ECA unit,
every timing receiver includes a TLU.
For each input connected to the TLU, 
there is a timestamp queue.
The rising or falling edge of a signal on the input
causes an absolute timestamp to be recorded in the 
respective queue.
Currently these timestamps are only accurate to 8ns,
though we intend to improve this to better than 1ns.

Although different form factors provide different physical control interfaces, 
we have unified all of them to a single C library interface.
A user of this library does not need to care how the timing 
receiver is attached to his computer.
He must only specify an appropriate address;
for example,
dev/wbm0 for PCIe or VME,
dev/ttyUSB0 for USB,
udp/192.168.100.100 for Ethernet.
Using this library, all devices in the timing receiver
can be automatically discovered using the self-describing bus standard.
Thus, one does not need to know the particular SoC address layout
for a given timing receiver.
Instead, one simply locates the component to control,
complains if it is missing,
and then proceeds to access it via the C interface to Wishbone.
This means, for example, 
that we can program the flash of all our form factors using the same 
software tool regardless of the physical interface.

Although the master interfaces for each form factor differ,
they all include a network connection in order to run White Rabbit.
It is possible to control the Wishbone bus of a timing receiver over
the network using the Etherbone protocol.
As with all other master interfaces to the SoC,
access proceeds via the same library calls.
Etherbone is simply a serial version of the Wishbone bus protocol.
We use this same protocol in the USB master interface.
Due to this network connectivity,
it is theoretically possible (though perhaps unwise)
to broadcast a firmware update to all timing receivers simultaneously.

\textbf{SCU}
Most timing receivers will be built in the 
Scalable Control Unit (SCU) form factor~\cite{scu}.
It is planned to run around 1200 units in FAIR.
The SCU is a combination of a carrier board with an Arria II FPGA and a
COMExpress board running Linux.
The communication between FPGA and COMExpress board is done via PCIe.
The carrier board connects the COMExpress Atom processor with
an Ethernet port, two USB ports, and a serial console.
The onboard FPGA is connected to two SFP slots, two LEMOs, DDR3, 
parallel flash, an LCD, and a serial console.
An SCU controls up to 12 slave cards via the SCU bus
and can connect to an optional daughter board for additional IOs.
The main use cases in FAIR for the SCU is the control of the ACU
(Adaptive Control Unit) for ramped power suplies,
control of radio frequency devices with FIB (FPGA Interface Board) and
to serve as a gateway to existing control system components connected via a 
bus derived from a MIL-STD-1553.

\textbf{PEXARIA}
The pexaria5 is a 4-lane PCIe card intended to be used in a standard PC.
It is based on the newer Arria V FPGA platform.
It sports up to four SFP cages, a 26-pin trigger bus interface,
a USB port, and an internal LCD.
In the future it will also host a daughter board with external IO
ports.
It is foreseen to be used for the DM and inin data acquisition for experiments
and beam instrumentation.

\textbf{EXPLODER}
The exploder is a portable timing receiver with many IO ports.
It is a two-planed, enclosed, hand-sized device 
hosting an Arria II FPGA and four SFPs.
The top plane contains the application-specific IOs,
such as LCD display, LEMOs, LVDS, NIM, trigger bus, knobs, etc.
The only additional master interface it provides is USB.
It is intended to be used in any situation where a hosted device,
such as the VME/PEXARIA/SCU would not be available.

\textbf{VETAR}
The vetar is a VMEbus IEEE-1014 card intended to be hosted in VME crates.
It also based on the Arria II FPGA and is endowed with one SFP cage, SRAM, LCD
display, EEPROM memory and USB interface. LVDS and lemo I/Os are available on the main 
board or on the extension board with additional high density connector.

\subsection{Timing Network}

The Timing Network interconnects the above described components of the Timing
System using WR switches~\cite{wr_switch}. A stable and continuous synchronization of all the
TRNs with an appropriate accuracy and reliable and deterministic distribution 
of timing messages are the key requirements. 
Both timing and data resilience against network failures is achieved 
using redundant connections. Lower layer protocols
establish a spanning tree topology setting ports connected to
cyclic paths to block/passive state. Upper-bound delivery latency of timing messages 
from the DM to TRNs is guaranteed using a QoS tagging~\cite{internet} of the
traffic and Cut-through~\cite{switch} switching.


\section{DEVELOPMENT AND IMPLEMENTATION PLAN}

The new challenge for building the FAIR timing system is
twofold: combination of already consolidate technologies like WR and 
Etherbone, and to create a reliable system capable of scaling
to more than 2000 nodes. 

The iterative development methodology is based on the creation of a relative simple GMT 
with minimal features and then increase the complexity and functionality, going
through iterations that will result in a running system.

\subsection{Proton Linac Source}

A complete control system will  be delivered for testing the source of
the proton linac, that will later serve as an additional injector into
the  existing  synchrotron  SIS18  at  GSI/FAIR.   In  summer  2013  a
functional prototype of this control system has been set-up and passed
its acceptance test.  From the  point of the timing system, this small
control  system is a  simple pulse  generator for  a handful  of TRNs.
However,  this  is  the  first  productive  timing  system  that  uses
prototypes  of the  FAIR timing  system, such  as a  timing  master, a
timing network and timing receivers.

\subsection{CRYRING}

 The next  step will be a  timing system
 for the CRYRING, a small  synchrotron and storage ring, which has been
 reallocated   from  the   Manne  Siegbahn   Laboratory   in  Stockholm
 \cite{cryring} to GSI  in 2013. It is presently  being set up next
 to the Experimental Storage Ring (ESR).  One of the motivations behind
 the CRYRING project is its  explicit usage for FAIR development tests.
 Although the operation of the  ring will commence in stand-alone mode,
 it  covers nearly  all relevant  aspects of  an  accelerator facility.
 Thus,  it presents an  ideal test  ground for  new sub-systems  of the
 accelerator  control system  such  as the  timing  system.  Here,  new
 development and features  for FAIR can be tested  without the overhead
 required for routine  operation.  It is estimated that  between 20 and
 50  TRNs have  to be  connected to  a first  timing  system.  Although
 limited in features, deployed components for the CRYRING timing system
 will  be close  to the  final ones  for FAIR.  A commissioning  of the
 timing system for CRYRING is planned for the first months of 2014.


\subsection{New Timing System for SIS18 and ESR}

 The timing systems for the proton linac source or  CRYRING have been 
 either very small in scale  our operated  under experimental  conditions.   
 Integrating the existing  timing  system  at  the  synchrotron SIS18  is  
 a  important milestone, since SIS18 serves as  an injector into the new FAIR machines.
 Operation  at SIS18  requires the  timing system  to work  reliably in
 routine operation 24/7.   The old timing system, which  is based on an
 extension of  the MIL-STD-1553  bus, needs to  be replaced by  the new
 GMT:  From the top-level  view of  the control  system stack  a common
 solution for  the timing system for  all ring machines  is required to
 guarantee and efficient transfer of ion bunches from one ring machine
 to the next. However, not all components of the existing timing system
 at SIS18 can be replaced and a SCU will be used as gateway.

\subsection{Integration of UNILAC}

Next to   the   new proton   linac,  the existing     heavy ion linear
accelerator UNILAC  is the second injector  into the  FAIR accelerator
complex.  Here, the existing MIL-based timing  system will most likely
not be  replaced for the reasons   of cost and effort.   Moreover, the
UNILAC is special  with respect  to  timing. First, it  is operated at
50Hz and phase locked to the mains voltage  delivered to GSI.  Second,
the ion sources feeding the UNILAC require  fixed repetition rates for
reasons such as thermal stability.  However, injecting the UNILAC beam
into the SIS18 requires linking  the UNILAC timing  to the schedule of
the FAIR accelerator complex.  This must  be achieved and tested prior
to commissioning the FAIR facility.


\subsection{FAIR Timing System}

All of the previous instances of
the new GMT had different aspects: Proof of principle at
the proton linac source, prototyping and development of
final solutions at the CRYRING and reliability of routine
operation at the SIS18 and the ESR. One important step 
towards the final timing system is the combination of all these
aspects. This is eased, since the timing master and the first
two layers of switches will be physically located in an existing 
building close to the SIS18. This allows to set up first
components of the timing master for CRYRING, SIS18 and
ESR already. By this, the timing master, its infrastructure
and its interfaces will be continuously developed, improved
and tested at its final location over a few years.
Another important step is to address the issue of scaling
the new GMT to the size required for the final FAIR facility. 
As it is planned to purchase the equipment not just
before FAIR machine commissioning but over a period of
several years, scalability can be addressed and tested in several steps. 
One option would be to use test areas in buildings of the existing facility.


\section{CONCLUSION}

A new general machine timing system for FAIR is presently being developed in
iteration cycles, where iteration provides a functional timing system
focusing in certain aspect of the final timing system. 
The next iterations  will implement timing system  for the source of the FAIR proton linac, 
followed by the CRYRING at GSI. The replacement of the existing timing 
system at the SIS18 and the ESR combined with addressing scalability will 
pave the way to the timing system for the FAIR facility.

\section{ACKNOWLEDGMENT}
The authors gratefully acknowledge the creativity and
support of the CERN White Rabbit Team, the driving force
behind the development of White Rabbit PTP. Furthermore
the authors acknowledge the help of the department of 
Experiment Electronics at GSI, who provided us with hardware modules and consulting.


\begin{thebibliography}{9}   % Use for  1-9  references
%\begin{thebibliography}{99} % Use for 10-99 references

\bibitem{fair_rep} T. Fleck et al., FAIR Accelerator Control System Baseline
Technical Report, General Machine Timing System, 2012.

\bibitem{gmt}D. Beck et al., F-DS-C-05e, FAIR Detailed Specification
“General Machine Timing System”, 2012.

\bibitem{interlock} J. Fitzek et al., F-DS-C-08e, FAIR Detailed Specification
“Interlock System”.

\bibitem{wr}White Rabbit \texttt{http://www.ohwr.org/projects/white-rabbit}.

\bibitem{internet} IEEE Std. 802.3-2008, 2008.
 
\bibitem{sync}Timing characteristics of a synchronous Ethernet equipment slave clock 
(EEC), ITU-T Std. G.8262, 2007.
 
\bibitem{ptp}IEEE Standard for a Precision Clock Synchronization Protocol for
  Networked Measurement and Control Systems, IEEE Std. 1588-2008,
  2008.
\bibitem{bootp} Bootstrap Protocol (bootp), RF 951, B. Croft J. Gilmore.

\bibitem{snmp} Simple Network Management Protocol, RFC 3411-3418.

\bibitem{sflow} sFlow, \texttt{http://wwww.sflow.org}.

\bibitem{dm} The Timing Master for the FAIR Accelerator Facility, B{\"a}r, T.
Fleck, M. Kreider, S. Mauro. ICALEPCS 11.  

\bibitem{scu} Performance of Standar FAIR Equipment Controller Prototype, Stefan Rauch

\bibitem{wr_switch}White Rabbit Switch  \\
\texttt{http://www.ohwr.org/projects/white-rabbit/wiki/Switch}

\bibitem{switch} The All-New Switch Book, Rich Seifert, James Edwards, Wiley Publsing 2010

\bibitem{cryring} M. Lestinsky et al, “CRYRING@ESR:A study group report”, \texttt{http://www.gsi.de Darmstadt, July 26, 2012}


\end{thebibliography}

\end{document}

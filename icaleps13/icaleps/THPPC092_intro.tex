\section{INTRODUCTION}

The GMT triggers and synchronizes accelerator equipment accordingly 
to the accelerator cycles~\cite{fair_rep}. Cycle lengths range 
from 20 ms (present UNILAC), several seconds (synchrotrons SIS18 
and SIS100/300) to several hours (storage rings)~\cite{gmt}. The beam production chain is an important concept in
accelerator control systems. It describes the production of a beam
from an ion source through the accelerators to a target. Properties of such a beam production 
chain include the ion type (from protons to uranium), energy, intensity, focus
and emittance and other parameters at the final destination. This information is
conveyed to the GMT, which 
has an integral view on the tightly synchronized accelerators and beam transfer
sections. The GMT must take into account the execution of several beam production 
chains in the accelerator complex at the same time. For each part of the machine, switching 
between different beam production chains will be possible between cycles, which implies a 
high degree of true parallel operation.

The GMT is made of an interconnected network of Timing Receiver Nodes (TRN) and
special nodes so-called Masters. The TRN are devices synchronized to the Timing
Master, source of time and frequency for all the network, and they are also the
receivers of timing messages from the Data Master. The interconnection is
established using WR Switches, which are responsible of the propagation of the
synchronization and timing messages to the TRN.

In  about  four years,  first  beam will  be  injected  into the  FAIR
accelerator complex. By then,  the General Machine Timing system
must support about 2000  Front End Controllers (FEC) with integrated TRN. 
The main tasks of the GMT are distribution 
of  timing messages, clocks  and timestamps.  Timing messages  synchronize actions of
the accelerator in hard real-time on a level of about 1 ns, even if
components are a few kilometers apart. Clock and timestamps
distribution, inherently  provided by White  Rabbit PTP, will  be used
for  correlating  data  acquired  at distinct  places  throughout  the
facility. 

The GMT  is linked to other  systems.  It must react  with upper bound
latency  on  external  signals  like interlock  signals~\cite{interlock}  by  executing
predefined alternatives in the  schedule. All systems connected to the
timing  system  depend on  it's  high  availability.  Distribution  of
timing messages, clocks and timestamps  must be guaranteed for commissioning and
testing even when the accelerator  does not produce beam. As a failure
in synchronizing equipment  may led to loss of  beam, the distribution
of timing messages must be robust -  at most one timing messages per year
may  be lost.  Furthermore, critical  components  of the  GMT must  be
implemented redundantly.

The  GMT not only  serves for  the operation  of the  FAIR accelerator
machines. Also Data AcQuisition (DAQ) systems of experiments will link
to  the  GMT  for   correlating  experimental  data  with  accelerator
actions. Moreover, a connection to the GMT allows for time stamping of
data in case of globally triggered DAQ systems, fan-out of time stamps
for  free running  DAQ systems  and finally  provides  distribution of
precise synchronous clock signals.

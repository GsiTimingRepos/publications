\section{INTRODUCTION}

The GMT triggers and synchronizes accelerator equipment accordingly 
to the accelerator cycles~\cite{fair_rep}. Cycle lengths range 
from 20 ms (present UNILAC), several seconds (synchrotrons SIS18 
and SIS100/300) to several hours (storage rings). The beam production chain is an important concept in
accelerator control systems. It describes the production of a beam
from an ion source through the accelerators to a target. Properties of such a beam production 
chain include the ion type (from protons to uranium), energy, intensity, focus
and emittance at the final destination and other parameters. This information is
conveyed to the GMT, which 
has an integral view on the tightly synchronized accelerators and beam transfer
sections. The GMT must take into account the execution of several beam production 
chains in the accelerator complex at the same time. For each part of the machine, switching 
between different beam production chains will be possible between cycles, which implies a 
high degree of true parallel operation.

The GTM is made of an interconnected network of Timing Receiver Nodes (TRN) and
special nodes so-called Masters. The TRN are devices synchronized to the Timing
Master, source of time and frequency for all the network, and they are also the
receivers of timing events from the Data Master. The interconnection is
established using WR Switches, which are responsible of the propagation of the
synchronization and timing events to the TRN.


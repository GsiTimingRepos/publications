\section{INTRODUCTION}

The GMT triggers and synchronizes accelerator equipment, 
timed according to the accelerator cycles [4]. Cycle lengths range 
from 20 ms (present UNILAC), several seconds (synchrotrons SIS18 
and SIS100/300) to several hours (storage rings). 

An important concept of the accelerator control system 
is the one of the so called beam production chain, which describes the production of a beam
from an ion source through the accelerators to a target. Properties of such a beam production 
chain include the ion type (from protons to uranium), energy, intensity, focus
and emittance at the final destination and other parameters. This is mapped to the GMT, which 
has an integral view on the tightly synchronized accelerators and beam transfer
sections. The GMT must take into account the execution of several beam production 
chains in the accelerator complex at the same time. For each part of the machine, switching 
between different beam production chains will be possible between cycles, which implies a 
high degree of true parallel operation.

For all components, set values or ramp data for the different beam production chains are 
preloaded in the Time Receiver Nodes (TRNs). This is done by the control system
via its general purpose network. The task of the GMT is to trigger and synchronize the 
TRNs in real-time via timing events, which carry additional information like machine ID
or event numbers. After parameterization of the facility by the top and middle layer of 
the control system, it is finally the GMT which autonomously operates the accelerator 
complex in real-time.




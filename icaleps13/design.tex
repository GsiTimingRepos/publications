\section{DESIGN}

Introduction to the chapter

\subsection{White Rabbit}

White Rabbit (WR~\cite{wr}) is a project which aims at creating
an Ethernet-based network with low-latency, deterministic
packet delivery and network-wide, transparent, high-accuracy
timing distribution. The White Rabbit Network (WRN) is
based on existing standards: Ethernet (IEEE 802.3~\cite{internet}),
Synchronous Ethernet (SyncE~\cite{sync}) and PTP~\cite{ptp}. 

A WRN consists of White Rabbit Nodes (nodes) and White
Rabbit Switches (switches) interconnected by fiber optics links. 
A node is considered the source and destination of information sent 
over the WRN. The information distributed over a WRN includes:

\begin{itemize}
    \item Timing - frequency and International Atomic Time.
    \item Data - Ethernet traffic between nodes.
\end{itemize}

\textit{to be finished}

\subsection{Timing System}

\subsubsection{Data Master}

Mathias

\subsubsection{Timing Master}

Cesar

\subsubsection{Management Master}

Marcus

\subsubsection{Timing Network}

Cesar

\subsubsection{Timing Receiver Nodes}

Stephan/Wesley
% form factors
% common design = SoC
% eca, tlu
% etherbone slave ==> 
% uniform access from pcie/vme/usb/ethernet


\section{DESIGN}

Introduction to the chapter

\subsection{White Rabbit}

White Rabbit (WR~\cite{wr}) is a project which aims at creating
an Ethernet-based network with low-latency, deterministic
packet delivery and network-wide, transparent, high-accuracy
timing distribution. The White Rabbit Network (WRN) is
based on existing standards: Ethernet (IEEE 802.3~\cite{internet}),
Synchronous Ethernet (SyncE~\cite{sync}) and PTP~\cite{ptp}. 

A WRN consists of White Rabbit Nodes (nodes) and White
Rabbit Switches (switches) interconnected by fiber optics links. 
A node is considered the source and destination of information sent 
over the WRN. The information distributed over a WRN includes:

\begin{itemize}
    \item Timing - frequency and International Atomic Time.
    \item Data - Ethernet traffic between nodes.
\end{itemize}

\textit{to be finished}

\subsection{Timing System}

\subsubsection{Data Master}

Mathias

\subsubsection{Timing Master}

Cesar

\subsubsection{Management Master}

At startup, the WR networking devices need essential configuration parameters (e.g. ip
address). During operation, the network has to be constantly monitored in
order to prevent, identify and solve malfunctions or breakdowns of devices. 
These tasks are carried out by specialized software (e.g DHCP) installed in the
so-called WR Management Master (WR MM). 

In the current WRN the MM is also a gateway between the corporative, timing and management 
network. All the WR switches are connected either to management and timing
network, while the WR nodes are only to the timing network. 

Currently the MM is serving ip addresses to the management port of switches and WR
nodes using BOOTP~\cite{bootp}. Information about the status of the network (e.g.
link up/down, synchronization status etc...)is gathered in the MM using Distributed Information Managment
tool. Finally, the WR Switches can boot using a NFS server in the MM for testing
purposes. 

In the future the MM will offer advance management and monitoring capabilities
using the SNMP~\cite{snmp} and sFlow~\cite{sflow}. These tools will allow to the 
WR network manager to anticipate networks problems and maintain the reliability
and robustness required for the timing system.


\subsubsection{Timing Network}

Cesar

\subsubsection{Timing Receiver Nodes}

Stephan/Wesley
% form factors
% common design = SoC
% eca, tlu
% etherbone slave ==> 
% uniform access from pcie/vme/usb/ethernet


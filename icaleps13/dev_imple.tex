\section{DEVELOPMENT AND IMPLEMENTATION PLAN}

The new challenge for building the FAIR timing system is
twofold: combination of already consolidate technologies like WR and 
Etherbone, and to create a reliable system capable of scaling
to more than 2000 nodes. 

The development strategy is based on the creation of a relative simple GMT 
with minimal features and then increase the complexity and functionality, going
through iterations that will result in a running system.

\subsection{Proton Linac Source}

Proton Linac Source In summer 2013 GSI will deliver a complete control 
system for testing the source of the proton linac, that will 
later serve as an additional injector
into the existing synchrotron SIS18 at GSI/FAIR. At this
stage, the timing system will only be a simple pulse generator 
for a handful of FECs. However, this will be the first
productive timing system that uses prototypes of the FAIR
timing system, such as a timing master, a timing network
and timing receivers.


\subsection{CRYRING}

CRYRING The next step will be a timing system for
the CRYRING, a small synchrotron and storage ring lo-
cated at the Manne Siegbahn Laboratory in Stockholm
[11]. This ring will be moved to GSI, where it will be set-
up next to the Experimental Storage Ring (ESR). One of
the motivations behind the CRYRING project is its explicit
usage for FAIR development tests. Although the operation
of the ring will commence in stand-alone mode, it covers
nearly all relevant aspects of an accelerator facility. Thus,
it presents an ideal test ground for new sub-systems of the
accelerator control system such as the timing system. Here,
new development and features for FAIR can be tested with-
out the overhead required for routine operation. It is esti-
mated that between 20 and 50 FECs have to be connected
to a first timing system. Although limited in features, de-
ployed components for the CRYRING timing system will
be close to the final ones for FAIR.

\subsection{New Timing System for SIS18 and ESR}

New Timing System for SIS18 and ESR The timing systems for 
the proton linac source or CRYRING have
been either very small in scale our operated under experimental conditions. 
Replacing the existing timing system
at the synchrotron SIS18 and the storage ring ESR is a important 
milestone, since the ongoing experimental program
requires the timing system to work reliably in routine operation 24/7. 
The old timing system, which is based on an extension of the MIL-STD-1553 bus, 
needs to be replaced by the new GMT: A common solution for the timing system 
for all ring machines should be used to guarantee and efficient transfer of 
ion bunches from one ring machine to the next.
 
\subsection{FAIR Timing System}
Final Timing System All of the previous instances of
the new GMT had different aspects: Proof of principle at
the proton linac source, prototyping and development of
final solutions at the CRYRING and reliability of routine
operation at the SIS18 and the ESR. One important step to-
wards the final timing system is the combination of all these
aspects. This is eased, since the timing master and the first
two layers of switches will be physically located in an existing 
building close to the SIS18. This allows to set up first
components of the timing master for CRYRING, SIS18 and
ESR already. By this, the timing master, its infrastructure
and its interfaces will be continuously developed, improved
and tested at its final location over a few years.
Another important step is to address the issue of scaling
the new GMT to the size required for the final FAIR facility. 
As it is planned to purchase the equipment not just
before FAIR machine commissioning but over a period of
several years, scalability can be addressed and tested in several steps. 
One option would be to use test areas in buildings of the existing facility.


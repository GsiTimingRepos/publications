\section{DEVELOPMENT AND IMPLEMENTATION PLAN}

The new challenge for building the FAIR timing system is
twofold: combination of already consolidate technologies like WR and 
Etherbone, and to create a reliable system capable of scaling
to more than 2000 nodes. 

The development strategy is based on the creation of a relative simple GMT 
with minimal features and then increase the complexity and functionality, going
through iterations that will result in a running system.

\subsection{Proton Linac Source}

A complete control system will  be delivered for testing the source of
the proton linac, that will later serve as an additional injector into
the  existing  synchrotron  SIS18  at  GSI/FAIR.   In  summer  2013  a
functional prototype of this control system has been set-up and passed
its acceptance test.  From the  point of the timing system, this small
control  system is a  simple pulse  generator for  a handful  of TRNs.
However,  this  is  the  first  productive  timing  system  that  uses
prototypes  of the  FAIR timing  system, such  as a  timing  master, a
timing network and timing receivers.

\subsection{CRYRING}

 The next  step will be a  timing system
 for the CRYRING, a small  synchrotron and storage ring, which has been
 reallocated   from  the   Manne  Siegbahn   Laboratory   in  Stockholm
 \cite{cryring} to GSI  in 2013. It is presently  being set up next
 to the Experimental Storage Ring (ESR).  One of the motivations behind
 the CRYRING project is its  explicit usage for FAIR development tests.
 Although the operation of the  ring will commence in stand-alone mode,
 it  covers nearly  all relevant  aspects of  an  accelerator facility.
 Thus,  it presents an  ideal test  ground for  new sub-systems  of the
 accelerator  control system  such  as the  timing  system.  Here,  new
 development and features  for FAIR can be tested  without the overhead
 required for routine  operation.  It is estimated that  between 20 and
 50  TRNs have  to be  connected to  a first  timing  system.  Although
 limited in features, deployed components for the CRYRING timing system
 will  be close  to the  final ones  for FAIR.  A commissioning  of the
 timing system for CRYRING is planned for the first months of 2014.


\subsection{New Timing System for SIS18 and ESR}

 The timing systems for the proton linac source or  CRYRING have been 
 either very small in scale  our operated  under experimental  conditions.   
 Integrating the existing  timing  system  at  the  synchrotron SIS18  is  
 a  important milestone, since SIS18 serves as  an injector into the new FAIR machines.
 Operation  at SIS18  requires the  timing system  to work  reliably in
 routine operation 24/7.   The old timing system, which  is based on an
 extension of  the MIL-STD-1553  bus, needs to  be replaced by  the new
 GMT:  From the top-level  view of  the control  system stack  a common
 solution for  the timing system for  all ring machines  is required to
 guarantee and efficient transfer of ion bunches from one ring machine
 to the next. However, not all components of the existing timing system
 at SIS18 can be replaced and a White Rabbit-MIL bridge is needed.

\subsection{Integration of UNILAC}

Next to   the   new proton   linac,  the existing     heavy ion linear
accelerator UNILAC  is the second injector  into the  FAIR accelerator
complex.  Here, the existing MIL-based timing  system will most likely
not be  replaced for the reasons   of cost and effort.   Moreover, the
UNILAC is special  with respect  to  timing. First, it  is operated at
50Hz and phase locked to the mains voltage  delivered to GSI.  Second,
the ion sources feeding the UNILAC require  fixed repetition rates for
reasons such as thermal stability.  However, injecting the UNILAC beam
into the SIS18 requires linking  the UNILAC timing  to the schedule of
the FAIR accelerator complex.  This must  be achieved and tested prior
to commissioning the FAIR facility.


\subsection{FAIR Timing System}

All of the previous instances of
the new GMT had different aspects: Proof of principle at
the proton linac source, prototyping and development of
final solutions at the CRYRING and reliability of routine
operation at the SIS18 and the ESR. One important step to-
wards the final timing system is the combination of all these
aspects. This is eased, since the timing master and the first
two layers of switches will be physically located in an existing 
building close to the SIS18. This allows to set up first
components of the timing master for CRYRING, SIS18 and
ESR already. By this, the timing master, its infrastructure
and its interfaces will be continuously developed, improved
and tested at its final location over a few years.
Another important step is to address the issue of scaling
the new GMT to the size required for the final FAIR facility. 
As it is planned to purchase the equipment not just
before FAIR machine commissioning but over a period of
several years, scalability can be addressed and tested in several steps. 
One option would be to use test areas in buildings of the existing facility.


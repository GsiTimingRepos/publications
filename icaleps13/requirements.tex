\section{REQUIREMENTS}


\textbf{From "THE NEW WHITE RABBIT BASED TIMING SYSTEM FOR THE FAIR
FACILITY" paper it has to be reviewed}

The requirements of the GMT have been described in the detailed specifications.
About 2000 FECs and other equipment are connected to the GMT, a distance of up to
2 km between the nodes has to be covered. In most of the cases a precision of 1
μs is sufficient. However, some equipment like kicker magnets for transferring bunches 
between machines require nanosecond precision.
Besides supporting control system features like equipment triggering and synchronization, 
parallel execution, varying machine cycle times and scalability, other key features are 
the following: \textit{Robustness} - At most one timing event per year may be
lost. \textit{Determinism} - Time critical information from must be distributed to the nodes with 
an upper bound latency. \textit{Redundancy} - Core components
of the GMT must be implemented with redundant equipment. \textit{Availability} - The GMT must be capable of 
distributing events for testing and commissioning equipment, even when the accelerator does not produce 
beam. \textit{Plan B Execution} - The GMT must react on external signals like interlock signals or malfunctioning 
equipment by executing predefined alternatives in the schedule. Other important features include 
integration of the existing machines and control system as well as interfaces to
many other subsystems like the interlock system and BuTiS.

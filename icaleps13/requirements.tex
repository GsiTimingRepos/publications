\section{REQUIREMENTS}

In  about  four years,  first  beam will  be  injected  into the  FAIR
accelerator complex. By then,  the General Machine Timing system
must support about 2000  Front-End Controllers and other timing
receivers. The main tasks of the GMT are distribution of  timing
events, clocks  and timestamps.  Timing events  synchronize actions of
the accelerator in hard real-time on a level of about 1 ns, even if
components are a few kilometers apart. Clock and timestamps
distribution, inherently  provided by White  Rabbit PTP, will  be used
for  correlating  data  acquired  at distinct  places  throughout  the
facility. The GMT must support true parallel operation of in different
parts of the  machine, as well as fast  switching between various beam
types (called multiplexing) for  each machine.  Typical cycle times in
the machine range from 20ms at  the present UNILAC up to several hours
in storage rings.

The GMT  is linked to other  systems.  It must react  with upper bound
latency  on  external  signals  like interlock  signals  by  executing
predefined alternatives in the  schedule. All systems connected to the
timing  system  depend on  it's  high  availability.  Distribution  of
events, clocks and timestamps  must be guaranteed for commissioning and
testing even when the accelerator  does not produce beam. As a failure
in synchronizing equipment  may led to loss of  beam, the distribution
of timing  events must be robust -  at most one timing  event per year
may  be lost.  Furthermore, critical  components  of the  GMT must  be
implemented redundantly.

The  GMT not only  serves for  the operation  of the  FAIR accelerator
machines. Also Data AcQuisition (DAQ) systems of experiments will link
to  the  GMT  for   correlating  experimental  data  with  accelerator
actions. Moreover, a connection to the GMT allows for time stamping of
data in case of globally triggered DAQ systems, fan-out of time stamps
for  free running  DAQ systems  and finally  provides  distribution of
precise synchronous clock signals.
